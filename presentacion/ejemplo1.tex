\documentclass[spanish]{beamer}
\usepackage[utf8]{inputenc}
\usepackage[spanish, es-ucroman, es-noquoting]{babel}
\usepackage{amsmath}
\usepackage{textpos} % Provee entorno textblock
\usepackage{mathdots} % Provee macro \iddots
\usepackage{circuitikz}


%%%%%%%%%%%%%%%%%%%%%%%%%%%%%%%%%%%%%%%%%%%%%%%%%%%%%%%%%%%%%%%%%%%%%%%%%%%%%%%%
% Modo handout (comentar para versión presentación en pantalla/proyector)
% \usepackage{pgfpages}
% \pgfpagesuselayout{4 on 1}[a4paper, landscape, border shrink=5mm]
% \setbeamertemplate{background canvas}{
%     \tikz \draw (current page.north west) rectangle (current page.south east);
% }
%%%%%%%%%%%%%%%%%%%%%%%%%%%%%%%%%%%%%%%%%%%%%%%%%%%%%%%%%%%%%%%%%%%%%%%%%%%%%%%%

% Quitar controles de navegación
\usenavigationsymbolstemplate{}

% Numerar las transparencias
\setbeamertemplate{footline}[frame number]

\title{Lógica Digital}
\subtitle{
  XXXXXXXX \\
  \vspace{2em}
  XXXXXX \\
  XXXX \\
  XXXXX
}
\author{XXX \\ \footnotesize{\texttt{<xxx@xxx.xx>}}}
\date{XX de Septiembre de XX}
\institute{
  Departamento de Computación \\
  Facultad de Ciencias Exactas y Naturales \\
  Universidad de Buenos Aires
}

\begin{document}
\begin{frame}
  \titlepage
\end{frame}

\begin{frame}
  \frametitle{Problema}
  \framesubtitle{Práctica 2, ejercicio 7 (parte A)}

  Dibujar el diagrama lógico de un \textit{codificador} de 4 líneas de
  entrada ($e_i$) y 2 líneas de saida ($s_i$). Si únicamente $e_i$ está alta,
  las salidas deben representar el número $i$ en notación sin signo. No está
  definido cuál es el resultado si no se cumple que sólo una de las líneas de
  entrada tiene valor 1.

  \begin{center}
    \vspace{1em}
    \includegraphics[width=5cm]{circuito.png}
  \end{center}
\end{frame}

\begin{frame}
  \frametitle{Tabla de verdad}

  \textit{``Si únicamente $e_i$ está alta, las salidas deben representar el
  número $i$ en notación sin signo.''}

  \pause

  \begin{center}
    \begin{tabular}{c|c|c|c||c|c}
      $e_0$ & $e_1$ & $e_2$ & $e_3$ & $s_0$ & $s_1$ \\
      \hline
      1 & 0 & 0 & 0 & 0 & 0 \\
      0 & 1 & 0 & 0 & 0 & 1 \\
      0 & 0 & 1 & 0 & 1 & 0 \\
      0 & 0 & 0 & 1 & 1 & 1 \\
    \end{tabular}
  \end{center}
\end{frame}

\begin{frame}
  \frametitle{Decisión de diseño}

  \textit{``No está definido cuál es el resultado si no se cumple que sólo una
  de las líneas de entrada tiene valor 1.''}

  \pause

  \vspace{1em}

  Una posibilidad podría ser bajar las líneas $s_0$ y $s_1$ cuando la cantidad
  de líneas de entrada altas es distinta de 1.

  \pause

  \begin{center}
    \begin{tabular}{c|c|c|c||c|c}
      $e_0$ & $e_1$ & $e_2$ & $e_3$ & $s_0$ & $s_1$ \\
      \hline
      1 & 0 & 0 & 0 & 0 & 0 \\
      0 & 1 & 0 & 0 & 0 & 1 \\
      0 & 0 & 1 & 0 & 1 & 0 \\
      0 & 0 & 0 & 1 & 1 & 1 \\
      \hline
      \multicolumn{4}{c||}{cualquier otra} & 0 & 0 \\
    \end{tabular}
  \end{center}
\end{frame}

\begin{frame}
  \frametitle{Suma de productos}

  \begin{columns}
    \column{0.5\textwidth}

    \begin{center}
      \begin{tabular}{c|c|c|c||c|c}
        $e_0$ & $e_1$ & $e_2$ & $e_3$ & $s_0$ & $s_1$ \\
        \hline
        1 & 0 & 0 & 0 & 0 & 0 \\
        0 & 1 & 0 & 0 & 0 & 1 \\
        0 & 0 & 1 & 0 & 1 & 0 \\
        0 & 0 & 0 & 1 & 1 & 1 \\
        \hline
        \multicolumn{4}{c||}{cualquier otra} & 0 & 0 \\
      \end{tabular}
    \end{center}

    \pause

    \column{0.5\textwidth}

    $s_0 = (\bar{e_0} \cdot \bar{e_1} \cdot e_2 \cdot \bar{e_3}) +
           (\bar{e_0} \cdot \bar{e_1} \cdot \bar{e_2} \cdot e_3)$

    \pause

    $s_1 = (\bar{e_0} \cdot e_1 \cdot \bar{e_2} \cdot \bar{e_3}) +
           (\bar{e_0} \cdot \bar{e_1} \cdot \bar{e_2} \cdot e_3)$


  \end{columns}

\end{frame}


\begin{frame}
  \frametitle{Circuito}

  \begin{center}
    % Código fuente para el diagrama de la solución: archivo solucion.xml.
    % Abrirlo desde https://www.draw.io.
    \includegraphics[width=10cm]{solucion.png}
  \end{center}
\end{frame}

\begin{frame}
  \begin{center}
    \Huge{¿Preguntas?}
  \end{center}
\end{frame}
\end{document}
